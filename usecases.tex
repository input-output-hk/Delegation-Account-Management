\section{Usecases}

In this section we describe various scenarios in which a user is expected to interact with the system. We describe the appropriate type of account for each case and the type of delegation that should be used and investigate the pros and cons of each choice.

\subsection{Bootstrapping a wallet}

Bootstrapping is the process of setting up a wallet that initially contains no money.

A user creates a base account $BA$. She also creates a certificate $C$ that delegates from $K^s$, the key used for the creation of $BA$, to a user-controlled key ${K^s}'$.

When she is about to receive her first money, the user gives the sender both the account address $BA$ and the certificate $C$. Therefore, when the sender creates a transaction that sends some money to $BA$, he also includes $C$ in the transaction data.

Therefore, the stake in $BA$ is controlled by $K^s$ and the user-created certificate $C$ has been published, so that the user can point to it from his pointer addresses.

\subsection{Delegating to a stake pool}

In order to delegate to a stake pool, a user needs to know the stake pool's staking key. After that, she points her pointer accounts to a transaction that contains a heavyweight certificate defining the stake pool's key as the delegate key and the chain rule as 0. She can either create such a transaction from an account that she owns or find an existing one on the blockchain.

\subsection{Cold staking}

Cold staking refers to the ability of using a ``hot" and a ``cold" key pair. The cold key exists offline in a secure location, whereas the hot key is online and is used for protocol-related functions, like issuing blocks. The cold key however is in control of the stake managed by the hot one, so that in case the hot key is compromised, the cold key can revoke it immediately and activate a new, uncompromised hot key.

\subsubsection{Stake pools}

The most usual case for cold staking is for stake pools. Specifically, the stake pool needs to be online at all times to issue blocks and take part in the MPC. However, their staking key might be target for attacks, so cold staking is a necessary tool to mitigate such attacks.

In this case, the stake pool issues a lightweight certificate using its public staking key, which is the key that users delegate to. The public staking key is the cold key, whereas the delegate key of the lightweight certificate is the hot one. The stake pool can then use the hot key and, when time comes to take part in the protocol, publish the lightweight certificate along with the block it generates or the MPC function it completes. In case of an attack, it can use the cold key to issue a new lightweight certificate to a new hot key, using a higher lightweight counter, so that the new certificate is immediately activated and the old lightweight certificate is effectively revoked.

\subsubsection{Users}

Suppose a user that wishes to keep her wallet offline, whilst changing her delegation profile frequently. That user can act as follows.

She keeps her balance in a base account $\alpha$ and creates a heavyweight certificate $c_0$ with chain rule $c = 1$ and delegate key $K_D = K^s_c$, which is her cold staking key. She then uses $K^s_c$ to create a heavyweight certificate $c_1$ with $c=1$ and $D = K^s_h$, her hot staking key. After that she can keep $K^s_c$ securely offline, whilst using $K^s_h$ to post a heavyweight certificate that delegates to some stake pool when she wishes to change her delegation profile. If $K^s_h$ is compromised and she wishes to revoke it, she uses $K^s_c$ to post a new heavyweight certificate to a new hot key ${K^s_h}'$.

\subsection{Changing the delegation profile}

A user might want to change the delegation profile for stake that she owns. In order to do that she must post a certificate using the staking key that controls her stake. However, that requires a transaction, that contains the certificate, to be posted on the blockchain. In order to do that, the user maintains an account online, which contains a small amount of money, enough just for posting a finite number of delegation certificates. We call such account a $pocket$ account. When the user wishes to post a certificate, she creates a transaction containing the certificate and sending all money in the $pocket$ account to a new ${pocket}_2$ account.

However, the user's stake might be spread over multiple accounts, which can be either online or offline. In that case the user can choose one of the following options, depending on the level of unlinkability she wants to maintain across her accounts and the fees she is willing to pay.

\subsubsection{Minimal fees}

The user issues a heavyweight certificate $c_0$ to a staking key $K^s$ with a chain rule $c = 1$. After $c_0$ is published, all accounts that the user creates point to it. $K^s$ is controlled by the user, so the user issues a heavyweight certificate $c_1$ with chain rule $c = 0$ to a stake pool of her choice. If she wishes to change her delegation profile, she simply posts a new heavyweight certificate ${c_1}'$ to a different stake pool.

This scheme allows the user to change her delegation profile by paying the fees for only a single heavyweight certificate. However, we should note that all accounts in her wallet will be linked together, since each account points to the same certificate.

\subsubsection{Unlinkability}

Unlinkability across accounts is a problem, given the fact that posting a certificate requires a transaction. Multiple accounts that are managed by the same staking key are linked together. Apart from that, posting certificates for multiple accounts using the same $pocket$ account also links them together.

A user that wishes to maintain the unlinkability across her accounts should use one staking key per account. Furthermore, she must use one $pocket$ account per stake account.

\subsection{Claiming delegation rewards}

In order to incentivize users to use delegation, the protocol gives rewards for delegating to stake pools. Specifically, when a stake pool issues a block, all accounts that have delegated to it receive some reward. However, in that case the delegation information for an account must be public.

A user can claim such rewards by simply using heavyweight delegation to a stake pool. Specifically, the stake in her account should link to a part of a delegation chain, the last certificate in which delegates to the stake pool's key.
